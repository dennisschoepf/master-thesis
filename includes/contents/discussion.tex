\section{Discussion}

With results from the study presented and summarized, it is now crucial to critically reflect on the approach, how these relate to other findings in the area and how releveant these findings are. All of this is subject of discussion within this chapter. To start of, the study approach is reflected upon.

% Discussion of results (justification phase): The critical discussion of the technical or analytical work steps is
% presented here and the relevance to innovation is established. Comparisons with other results on similar
% questions in the literature are entirely appropriate (reference to questions and knowledge gaps in existing
% literature on the topic identified in the “Related Work” chapter). Arguments may also be presented here
% concerning social relevance and responsibility with regard to the research question, or it may be placed it in
% the context of an overall social discourse.

\subsection{Reflection on the Study Approach}

This reflection includes reconsidering the methodological choices and research design, discussing the participant selection and overall study results and possible limitations of the study as a whole. In order to do this, the initial research goal has to be compared to what was researched in the end -- this allows for reconsideration of if the goals were achieved overall. As described in the introductory part of this thesis, this overall research goal was to take a first step into research play and playfulness within the onboarding process of software development projects. Due to this intersection of area not being researched, an exploratory step into it has been taken. This means that a broad range of ideas should be generated, but also the attitude of people within the targeted context should be researched. Both of these should be done in order to gather information about if the mentioned intersection of areas is even desired and secondly how such an intersection could be deployed in the context. Thus, multiple different approaches and aspects should be researched rather than quantifiably and objectively investigating a single aspect.

As mentioned before, a qualitative research approach can be considered sensible in this case, as it can be used to investigate beliefs and attitudes. It also can be valuable in investigating views on a focused topic, which is what is needed in this case \cite{10.1093/humrep/dev334}.

More importantly, the actual methods used have to be reflected on different accounts. Firstly, the the creation of the technology probe is critically observed and secondly, the results of the study is reflected on qualitative research quality criteria proposed by Flick \cite[p. 541-549]{flick2018introduction}.

The study design and process was largely defined by the fact that a \textit{technology probe} was used. A definitive upside of this approach that also influenced the outcome of the study was the fact that participants were directly confronted with a possible implementation of what was discovered in literature. This allowed for a more focused and structured data inquiry, which ultimately led to a large amount of data generated on the selected approaches as well as the overall attitude to playful helpers. A possible downside would be the fact that the participants could have been influenced by what they have already experienced and thus did not get the chance to freely think about possible other approaches to the topic. To gather valuable input nonetheless, the probe was rooted in already existing research on play. This allowed for an informed foundation of the probe, so that the participants answers could be set into the context of the guidelines and features crafted before.

Regarding the selection of features implemented in the probe, there could have been a myriad of other possible designs. Here some kind of pre-selection through either literature or initial feedback from a very small set of developers could have been valuable. It has to mentioned though, that implementing multiple different designs in separation, just to select a subset of features, would have been out of scope of this thesis. Rather the interactive part of the probe itself can be seen as a possible implementation of a subset, that can be expanded upon through future research.

Concerning the combination of qualitative, problem-focused questions, there are some lessons learned though. Rather than including the questions as a separate step after the interactive part of the probe, these questions could have been integrated more deeply. Out of the eight participants that finished the probe only five provided the full set of answers -- here well-integrated questions maybe brought forth by the companion could have led to more involvement and more questions answered by all participants. Certainly setting up the study in a remote and asynchronous manner could have decreased the involvement of the participants. As described before in this thesis, this was a compromise due to the current situation, during which on-site studies would have been hard to execute. Nonetheless, it was discovered through the study, that multiple participants could imagine themselves using tools similar to the probe for open-source projects, which often are organized remotely and asynchronously. Thus, the chosen approach was valuable in exploring how developers would react to tooling that reflects such an organization structure. Still, the study is therefore limited in significance towards aspects of onboarding present in on-site social settings. This presents an opportunity for future research into playfully enhancing those aspects specifically.

Concerning the quality of the study and analysis as a whole, there are multiple criteria the study can be discussed upon. For one, there is the topic of reliability. While this cannot be used similar to how it is used for quantitative studies, procedural reliability is an important aspect of qualitative studies. Regarding the data analysis, this can be increased through reflexive coding and contrasting results against each other. Another important aspect is to divide the data from its interpretation and by documenting the process in great detail \cite[p. 541-543]{flick2018introduction}.

In the case of this study, this is attempted through careful documentation of how the process was taken from literature, extended upon and documenting the decisions needed to arrive at the study setup. Further, all steps connected to data were documented in great detail, as can be seen in the respective appendices and the repository containing supplementary material. The coded data can be clearly traced back to its source with every transformation documented. Additionally, a two-step approach to coding as well as for developing the category system tries to further increase the reliability through incrementally adapting coding units and categories. This documentation also tries to increase the validity of the results through making the research process as transparent as possible \cite[p. 546]{flick2018introduction}. It has to be noted though, that the onboarding done within the technology probe does not fully reflect an actual onboarding setting, as it was specifically crafted as part of the probe. Researching that onboarding context in greater detail, would have required a more observational approach, though. However, this would have not allowed for investigating possible applications of playful themes from literature or inquiring developers on their attitude towards play in their context. This would have to be done in separate steps removed from observation of the setting, rather than doing it within a coherent probe, that provides broader results.

Overall, the study and preceding literature research achieved the goals set in the beginning. Due to its very broad setup, though, the results are limited in their specificity towards single aspects of the process. Thus, the prime motivation of future research should be in gaining more detailed knowledge on specific aspects of playful elements in onboarding processes -- e.g. through focusing on the social structures of projects and how they could be made more playfully. Additionally, the question of possibly increased effectiveness concerning the learning process in onboarding still has to be answered.

\subsection{Related Work}

At this point the results of the study have to be contrasted to the related work researched on in the beginning of this thesis. As this thesis tries to combine multiple different lines of research, this has to be done from multiple angles: The relation to other research on play and to software development with a focus on onboarding onto new projects.

Starting with the relation to play, multiple aspects of playful designs have been mentioned throughout literature research, three of which are reflected upon at this point. Firstly, there is the question of how the playful implementation during the study relates to other playful designs.

% -> Relation to what play is described at -> how playful is it really? -> Would have been a tried and tested game better as the implementation quality is not at question -> Plaful Probe

% -> Relation to other playful designs -> most similar to Bogost's educational tools and explorative case studies brought forth by Sicart e.g.

% -> Relation to gamification -> You could use progress reminders, badges, ... as part of the process to increase engagement !!! but overall they are still not what makes play enjoyable and therefore only should be used in combination with playful elements

% ----------------------
% software Devlopment / Onboarding:
% Yates -> not something that specifically tries to increase efficiency towards feature location and pulling information => Rather an educational tool
% Example with tool to answer questions -> Less focused on efficiency, rather on motivating through the process
% Enjoyment, developer happiness important -> could they increase that - Open for discussion

% ----------------------
% Overall, there is a place for this kind of approach in research -> As play is something that developers seem open towards -> CITE

% Ultimately in relation to other work this study opens up new possibilities not explored until now and simultaneously hopes to inform future research

\subsection{Relevancy \& Future Implications}

% Possible other implementations:
% 
% List here everything that was not done in the probe, but that came up as an idea and put it out there for future research ,e.g.
% 
% - Tell a story through git commits (e.g. text-adventure style as an additional medium)
% - Procedurally generated game world, where players can destroy parts of a project to see what breaks due to that
% 
% -> And every idea from the participants of the probes
% 
% 
% \blockquote{
%   \textit{Note for feedback}
%   \textbf{This is going to be in this part:}
% 
%   How relevant could all of the research be? Can we start to use it immediately? If not, what has to be done? How could we measure effectiveness in the future?
% 
% }