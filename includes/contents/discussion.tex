\section{Discussion}

With results from the study presented and summarized, it is now crucial to critically reflect on the approach, how these relate to other findings in the area and how relevant these findings are. All of this is subject of discussion within this chapter. To start off, the study approach is reflected upon.

\subsection{Reflection on the Study Approach}

This reflection includes reconsidering the methodological choices and research design, discussing the participant selection and overall study results, and possible limitations of the study as a whole. In order to do this, the initial research goal has to be compared to what was researched in the end -- this allows for reconsideration of if the goals were achieved overall. As described in the introductory part of this thesis, this overall research goal was to take a first step into research play and playfulness within the onboarding process of software development projects. Due to this intersection of areas not being researched, an exploratory step into it has been taken. This means that a broad range of ideas should be generated, but also the attitude of people within the targeted context should be researched. Both of these should be done in order to gather information about if the mentioned intersection of areas is even desired and secondly how such an intersection could be deployed in the context. Thus, multiple different approaches and aspects should be researched rather than quantifiably and objectively investigating a single aspect.

As mentioned before, a qualitative research approach can be considered sensible in this case, as it can be used to investigate beliefs and attitudes. It also can be valuable in investigating views on a focused topic, which is what is needed in this case \cite{10.1093/humrep/dev334}.

More importantly, the actual methods used have to be reflected on different accounts. Firstly, the creation of the technology probe is critically observed and secondly, the results of the study are reflected on qualitative research quality criteria proposed by Flick \cite[p. 541-549]{flick2018introduction}.

The study design and process was largely defined by the fact that a \textit{technology probe} was used. A definitive upside of this approach that also influenced the outcome of the study was the fact that participants were directly confronted with a possible implementation of what was discovered in literature. This allowed for a more focused and structured data inquiry, which ultimately led to a large amount of data generated on the selected approaches as well as the overall attitude to playful helpers. A possible downside would be the fact that the participants could have been influenced by what they have already experienced and thus did not get the chance to freely think about possible other approaches to the topic. To gather valuable input nonetheless, the probe was rooted in already existing research on play. This allowed for an informed foundation of the probe so that the participants' answers could be set into the context of the guidelines and features crafted before.

Regarding the selection of features implemented in the probe, there could have been a myriad of other possible designs. Here some kind of pre-selection through either literature or initial feedback from a very small set of developers could have been valuable. It has to be mentioned though, that implementing multiple different designs in separation, just to select a subset of features, would have been out of scope of this thesis. Rather the interactive part of the probe itself can be seen as a possible implementation of a subset, that can be expanded upon through future research.

Concerning the combination of qualitative, problem-focused questions, there are some lessons learned though. Rather than including the questions as a separate step after the interactive part of the probe, these questions could have been integrated more deeply. Out of the eight participants that finished the probe, only five provided the full set of answers -- here well-integrated questions brought forth by the companion could have led to more involvement and more questions answered by all participants. Certainly setting up the study in a remote and asynchronous manner could have decreased the involvement of the participants. As described before in this thesis, this was a compromise due to the current situation, during which on-site studies would have been hard to execute. Nonetheless, it was discovered through the study, that multiple participants could imagine themselves using tools similar to the probe for open-source projects, which often are organized remotely and asynchronously. Thus, the chosen approach was valuable in exploring how developers would react to tooling that reflects such an organizational structure. Still, the study is therefore limited in significance towards aspects of onboarding present in on-site social settings. This presents an opportunity for future research into playfully enhancing those aspects specifically.

Concerning the quality of the study and analysis as a whole, there are multiple criteria the study can be discussed upon. For one, there is the topic of reliability. While this cannot be used similarly to how it is used for quantitative studies, procedural reliability is an important aspect of qualitative studies. Regarding the data analysis, this can be increased through reflexive coding and contrasting results against each other. Another important aspect is to divide the data from its interpretation and by documenting the process in great detail \cite[p. 541-543]{flick2018introduction}.

In the case of this study, this is attempted through careful documentation of the process and the decisions needed to arrive at the study setup. Further, all steps connected to data were documented in great detail, as can be seen in the respective appendices and the repository containing supplementary material. The coded data can be traced back to its source with every transformation documented. Additionally, a two-step approach to coding as well as for developing the category system tries to further increase the reliability by incrementally adapting coding units and categories. This documentation also tries to increase the validity of the results through making the research process as transparent as possible \cite[p. 546]{flick2018introduction}. It has to be noted though, that the onboarding done within the technology probe does not fully reflect an actual onboarding setting, as it was specifically crafted as part of the probe. Researching that onboarding context in greater detail would have required a more observational approach, though. However, this would have not allowed for investigating possible applications of playful themes from literature or inquiring developers on their attitude towards play in their context. This would have to be done in separate steps removed from observation of the setting, rather than doing it within a coherent probe, that provides broader results.

Overall, the study and preceding literature research achieved the goals set in the beginning. Due to its very broad setup, though, the results are limited in their specificity towards single aspects of the process. Thus, the prime motivation of future research should be in gaining more detailed knowledge on specific aspects of playful elements in onboarding processes -- e.g. through focusing on the social structures of projects and how they could be made more playfully. Additionally, the question of possibly increased effectiveness concerning the learning process in onboarding still has to be answered.

\subsection{Related Work \& Relevancy}

At this point, the results of the study have to be contrasted to the related work researched at the beginning of this thesis. As this thesis tries to combine multiple different lines of research, this has to be done from multiple angles: The relation to other research on play and to software development with a focus on onboarding onto new projects.

Starting with the relation to play, multiple aspects of playful designs have been mentioned throughout literature research, three of which are reflected upon at this point. Firstly, there is the question of how the playful implementation during the study relates to other playful designs and play in general.

An attempt at answering that can be made by setting it into context of related work in play. Concerning, e.g. the attributes of play Sicart described, some can arguably be found in the implementation of the study while others might not. The contextuality of play for example \cite[p. 6]{sicart2014play} certainly played one of the most important parts in this implementation, by trying to use that contextual information for a goal that is removed from play as \enquote{a source of gratification or escape for its individual participants} \cite[p. 213]{wein2000suttonreview}. Concerning a theoretically similar approach, Bogost's examples on play and games as educational tools could be the closest to what was created in the case of this thesis \cite{bogost2007persuasive}.

What has to be noted here is that in itself the probe does not have to be a full-fledged game or attain to all attributes of play, as this is not its goal. Rather it should be seen as a tool used to investigate how to create these in the future. As mentioned before, it rather investigates a subset of implications derived from literature, to inform future designs -- possibly full-fledged games.

A question that can be asked in this case, is if using an existing game and just using contextual information would have provided other results. An approach for this would have been \textit{playful probing} as described by Bernhaupt et al., proposing adapting existing games to a context to explore it \cite{bernhaupt2007playful}. This certainly could have been another valid approach in the case of this thesis and eliminated a small amount of ambiguity towards the results as the specific game mechanic would have been tried and tested. Where such an approach falls short though is in creating playful features specifically targeted towards the context and investigating how literature can be translated into playful elements. To summarize, playful probing could be a valid research method concerning the overall research topic. Due to the research goals including the investigation of applied playful features, a custom implementation presented a better approach in this case.

Another related topic that was mentioned both in literature research and also during the study itself is \textit{gamification}. As there were valid criticisms of this approach and its limited view on play, a conscious decision on excluding points, badges, or leaderboards from the implementation was taken. This was done to create a playful activity rather than a gamified experience solely focusing on points and badges. Fully excluding them on the other hand, led to input from participants, that they would want to include some kind of competition (e.g. leaderboards) or to have set goals they can follow and achieve (and maybe gain a badge in the process), as discussed in the results chapter. A hybrid approach similar to how Knaving described it \cite{knaving2013designing}, might have yielded better results and should be part of future research and designs. Nonetheless, it is crucial to not only use these gamification mechanics but rather create a holistic playful activity possibly including some of those mechanics.

Regarding related work on software development onboarding, there are some additional aspects worth discussing. Firstly, the aspect of what the study approach wants to achieve in the onboarding context. As described by Yates there are multiple different levels from which onboarding can be looked at, with psychology-, process-, and developer-centered approaches. In addition to that Yates also describes different areas within onboarding, e.g. program comprehension as an area of interest or the idea of push- and pull-information \cite{yates2014onboarding}. Most of these aspects were already discussed at this point, therefore only a summary is given on the relation of what was discovered throughout this thesis: A push-based digital onboarding tool was created, aiming to support information seeking during onboarding processes in a playful way. This stands in contrast to many other technical approaches that either try to provide information in an efficient way \cite{dominic2020onboarding} or assist the developer in quickly navigating through code\footnote{e.g. the \textit{Peek} feature in Visual Studio code: \url{https://code.visualstudio.com/docs/editor/editingevolved}, accessed on 17th of August 2021}. The goals of such tools and research are very much aiming for an increase in efficiency. This approach differs here, the overall goal is not to develop a more efficient way of onboarding but rather a more enjoyable and ultimately more motivating one. Thus, the study results remain relevant in this line of research, even though -- or even because of -- this shift in focus. Regarding the immediate application of the approach in actual onboarding settings, though, the results of this study are less applicable. The relevance lies more in the results being able to inform future applications of playful elements in onboarding.
