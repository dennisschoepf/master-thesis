\section{Results}

With the study executed and analyzed -- using the methods described in the methdology chapter and following the blueprint laid out in the research design chapter -- the results can now be elaborated on in detail and connected to the research questions of this thesis. This elaboration starts with recapitulating and aggregating the final set of playful design implications discovered in literature. After that the results of the study are described in detail as well. This includes the final number of participants, a contextual description of the participants, and their qualitatively analyzed input supported with the answers on the underlying project and the data logged throughout the probe execution. Finally, these results are set into the context of the research questions, so that these can be answered with what was researched throughout the whole thesis.

\subsection{Playful Design Guidelines}

The foundation for the playful design guidelines were discovered through the state-of-the-art literature research and the literature research on the theoretical grounding of this thesis. The actual guidelines were ultimately created as the first step towards the probe implementation. An excerpt of these implications was mentioned before, and the full set of implications can be found in the respective diagram in \textit{Appendix \ref{append:probe-design}}. The final set of playful design implications therefore can be summarized as follows:

\begin{itemize}
  \item{Use serendipitous effects throughout the context}
  \item{Let the player create freely within a space}
  \item{Let the player change the context and visualize the effects}
  \item{Let the player manipulate and destroy parts of the context and visualize the effects}
  \item{Intentionally let the player see through the cracks of the game world}
  \item{Create a game world from contextual data}
  \item{Humanize the context and generate communication through it}
  \item{Create a stage for the player, where they can explore without too much guidance}
  \item{Use uncertainty \& do not reveal everything to the player}
  \item{Utilize playful elements already present in the context}
  \item{Allow for informal and personal communication with the player}
  \item{Purposefully create mysterious elements}
\end{itemize}

Some of these implications resemble very closely what was discovered in existing literature, some are aggregations of multiple sources done by me. For the specific literature this is based upon, refer to the respective diagram in \textit{Appendix \ref{append:probe-design}} and the reasoning in chapter 2.4. This set of design guidelines is one part of the contributions of this thesis as a whole. It informed the creation of the technology probe outlined in the research design chapter, but could also serve as guidelines for the future creation of other playful designs.

\subsection{Study Results}

With the recapitulation of these guidelines as the foundation of the technology probe, the focus can now shift towards the results of that technology probe. This probe is based on the methods mentioned and discussed about in the methodology chapter and was executed as described within the research design. At first there is going to be an overview of the final set of participants. This is enriched with some general aggregate data computed from logging the interactions during probe execution. After that the results of the interim goals during the qualitative content analysis are presented. Ultimately, an attempt at answering the research questions through interpreting and presenting the results of that analysis is made.

Before that can be done, as mentioned, the set of participants has to be presented. As described in both the chapter on methodology and research design, the only requirement towards the participants is that they are a software developer of some sorts. Therefore, two companies that employ software developers were initially contacted and asked to share the \gls{url} to the technology probe with their software developers. This resulted in five initial participants. As the resulting data was not sufficiently diverse, four additional developers were contacted directly via e-mail and also asked to share the probe with their colleagues. Ultimately, nine participants executed the technology probe, with eight of them finishing the interactive part and five finishing the interactive part as well as providing answers to all the questions. The participants were aged 24 to 36, and the participants professional experience varied from zero to more than 10 years. Professions of the participants also varied in detail, with web and software developers, data scientists and others. The final selection with the specific information on each participant and a participant code, that is going to be used going forward can be seen in \textit{Appendix \ref{append:participants}}.

Overall these eight participants went through the interactive part of the probe in times between 2 minutes 18 seconds and 12 minutes 50 seconds. With the exception of one outlier, around 120 to 180 interactions where needed to discover all the revealables generated for the probes. While these aggregate numbers do not carry much meaning by themselves, they (and the single interactions log data points they are based upon) can be used to add contextual information to the qualitative data gathered throughout the study. An overview of these aggregations for each participant can be found in \textit{Appendix \ref{append:log-aggregated}}.

Concerning the content analysis of the qualitative data, it was executed as described in chapter 3.4. In short, there are four phases to go through with the chosen approach: \textit{Selecting material, creating a category system and coding the content with the specified categories}. As mentioned earlier, selecting the material is trivial in the case of this study. Where additional work was done, is in developing a category system that then is used for coding the material, as described by Schreier \cite[p. 174]{schreier2014ways}. To develop such a system it is crucial to clarify what should be achieved by the analysis. In this case, the ultimate goal is to answer the research questions with the data that was generated. Thus, these categories should serve the goal of coding the material with the research questions in mind. This can be achieved by deductively creating the categories from the research questions and the literature they are based upon. The result of this procedure can be seen in \textit{Table \ref{tab:qa-categories}}.

\begin{table}[h]
  \begin{tabularx}{\textwidth}{| X | X |}
    \hline
    \textbf{Research Question}                                                                                                                                                      & \textbf{Categories}                                                                                                            \\ \hline
    (RQ1) How and which theories \& concepts from research on play can be used in the onboarding process of software development projects and how are developers experiencing them? & Ideas on playful elements for onboarding                                                                                       \\ \hline
    (RQ2) Which playful themes could lend themselves well to evolve software development projects into a spaces of play?                                                            & Experience towards playful probe features                                                                                      \\ \hline
    (RQ3) What are the technical, social and personal intricacies of an onboarding context in software development and how do they influence the onboarding experience?             & Social Onboarding (Aspects), Technical Onboarding (Aspects), Organizational Onboarding (Aspects), Ideas on Onboarding Elements \\ \hline
    (RQ4) How can the identified themes be applied in actual onboarding settings?                                                                                                   & Improvements for implemented features                                                                                          \\ \hline
    (RQ5) How do different kinds of software developers experience playful aspects within an onboarding context?                                                                    & Attitude towards Play in Onboarding, Contextual Considerations                                                                 \\ \hline
  \end{tabularx}
  \caption{\label{tab:qa-categories}Content Analysis -- Category System developed from the research questions}
\end{table}

For each of the research questions at least one category has been created, aiming to synthesize valuable data that can be used to answer these questions. If during the analysis of the coding categories were deemed not concise enough for important units of content, sub-categories were inductively created. More specifically, the category of \textit{ideas on onboarding elements} was added to create a distinction on feature-oriented input by the participants. Additionally, a category on \textit{contextual considerations} has been created, as there has been multiple instances input on how the approach on playfulness changes depending on the onboarding context. These categories as a whole then were used to code the content. Where these codes were not enough to create a meaningful unit of content, contextual units were used. As described in chapter 3.4, these contextual units can be supportive data from the logs as well as the playful features themselves. With that information distinct units of text can be coded and set into context and ultimately interpreted in relation to the research questions.

The actual coding process that came after, was two-fold. At first, units of text that contained important content was marked. As a second step, these units of text were then collected and coded using the set of categories defined beforehand. After creating the two additional, inductive categories, the units were examined again, to adapt categories where needed. The result of this process, with every single coded unit of text together with contextual units -- where applicable -- can be seen in \textit{Appendix \ref{append:coding}}. Throughout the results chapter, if a coding unit is referred to, it is done so by an auto-generated number also present in the table within the respective \textit{Appendix \ref{append:coding}}.

Together with all of the work that has been executed prior -- researching the origins of play, contemporary views, creating design guidelines, implementing features, creating a technology probe -- this finally allows for careful interpretation of everything that was gathered up until this point.

\subsubsection*{(RQ2) Which playful themes could lend themselves well to evolve software development projects into a spaces of play?}

Concerning this research question much of the effort has already been expended. As described in chapter 3.1, the playful themes were discovered through literature research and careful condensing of what was found into foundations for the playful design guidelines. Thus, this can be seen as the answer to the first part of this question. If they lend themselves well in the context of onboarding onto software development projects is unanswered, though. As was mentioned in chapter 3.4 as well, a generalizeable and reliable answer on effectiveness is out of scope in this thesis. Rather, initial feedback on the subset of playful features implemented has been gathered and can serve as guidance for future applications of play in onboarding processes.

Concerning the theme of creating an intentionally ambigious game world for example there were mixed irritated experiences reported by the participants. P1 mentioned here: \enquote{I was a bit confused because I didn't know what to do when I reached the first blank slide}\footnote{Coding Unit Nr. \textbf{1} in \textit{Appendix \ref{append:coding}}}. Regarding P1 this was also supported by the logged interactions, where the participant did not perform any action at all, until the companion showed the help message\footnote{see lines 10-63 in suppl. repository at path: \path{./supplementary-material/raw-probe-data/technology-probe-logs.json}}. Multiple other participants mentioned that as well. P3 also mentioned that it was not clear how the reveal process within the detail scene should begin and that he did not not know how to start in that scene\footnote{Coding Unit Nr. \textbf{44} in \textit{Appendix \ref{append:coding}}}, which also is supported by this participant's logs\footnote{see lines 9040-9090 in suppl. repository at path: \path{./supplementary-material/raw-probe-data/technology-probe-logs.json}}. P8 mentioned some problems with the detail scene as well, mentioning \enquote{I needed some time and the help box to understand, that I was supposed to click somewhere to reveal the information}\footnote{Coding Unit Nr. \textbf{91} in \textit{Appendix \ref{append:coding}}}. Overall it could be assumed, that the large amount of ambiguity, with completely hidden objects may have gone too far. This is even more important when using those elements as a way to gain knowledge. Here small indicators, or evolving objects that initially are shown in a game world, might have resulted in better understanding.

Here the companion feature was very helpful for the participants, though. P1 for example mentioned that \enquote{helper on the bottom was quick to assist, though}\footnote{Coding Unit Nr. \textbf{14} in \textit{Appendix \ref{append:coding}}} and that it \enquote{came to help always at the right time. Good guy!}\footnote{Coding Unit Nr. \textbf{17} in \textit{Appendix \ref{append:coding}}}. This was also mentioned by P3, albeit in a less personal tone\footnote{Coding Unit Nr. \textbf{45} in \textit{Appendix \ref{append:coding}}} and with the notion, that there was no interaction other than the help messages\footnote{Coding Unit Nr. \textbf{48} in \textit{Appendix \ref{append:coding}}} (this was also mentioned by P6\footnote{Coding Unit Nr. \textbf{71} in \textit{Appendix \ref{append:coding}}}). Overall though, the experience towards the companion was very positive and could be a theme worth exploring further in this context.

The reveal mechanic -- spawning bubbles to reveal information -- on the other hand was perceived more mixed. P8 mentioned that it was fun at first, but in the end it got repetitive\footnote{Coding Unit Nr. \textbf{93} in \textit{Appendix \ref{append:coding}}}. P8 also reported problems catching the revealable objects\footnote{Coding Unit Nr. \textbf{93} in \textit{Appendix \ref{append:coding}}}, which can be traced to the logs as well\footnote{see column \textit{Contextual Unit (Logs)} Nr. 93 \textit{Appendix \ref{append:coding}}}. Albeit the technology probe should be kept very simple according to the literature on it, additional or changing mechanics could have helped here.

Other important notations were targeted towards the overview scene. Both P8 and P6 mentioned that the randomly generated overview was pretty chaotic and therefore the structure of the subprojects were hard to see\footnote{Coding Units Nr. \textbf{64, 94} in \textit{Appendix \ref{append:coding}}}. A more structured visualization could have helped here, although P9 specifically pointed out, that it was helpful to see objects for all the subprojects\footnote{Coding Unit Nr. \textbf{116} in \textit{Appendix \ref{append:coding}}}. Mostly positive feedback was gathered towards the aspects of the underlying project\footnote{Coding Units Nr. \textbf{37, 38, 41, 42, 88, 118} in \textit{Appendix \ref{append:coding}}}. It has to be mentioned, though, that especially P1 had mixed feedback\footnote{Coding Units Nr. \textbf{4, 12} in \textit{Appendix \ref{append:coding}}}. Important to note here, might be that P1 -- in contrast to all other participants -- mentioned that contributors \enquote{don't really help me to get started with a project I guess}\footnote{Coding Unit Nr. \textbf{3} in \textit{Appendix \ref{append:coding}}}.

Overall the study reinforced some playful themes, especially the companion with personal communication and help towards the participants. Others, like the extremely ambigious initial detail scene, irritated more than it helped. Crucial to mention here is that with this study setup no definitive assertion can be made on the effectiveness of these features, it only explores personal subjective experiences of the participants -- to possibly inform such studies on subsets of playful themes in the future.

\subsubsection*{(RQ3) What are the technical, social and personal intricacies of an onboarding context in software development and how do they influence the onboarding experience?}

The approach to answering RQ3 is very similar to the previously discussed research question. Regarding this question, a considerable amount of literature exists and was discussed during the literature research part of this thesis. To summarize, there are different schools of approaches towards software development onboarding: Psychology-centered, process-centered \& developer-centered approaches \cite{yates2014onboarding}. These all focus on different aspects of the onboarding process. Technical intriciacies (psychology-centered approaches) can be program comprehension and programming (or similar purely technical) skills. Social -- and organizational -- intricacies include the project setup as a whole (remote vs. on-site, open-source vs. commercial), but also the project members, experiences of colleagues working on the same project, availability of colleagues or organizational and hierarchical structures behind a project. Lastly, personal intricacies are overlapping with both technical and social challenges. These include personal approaches and processes towards onboarding, like information seeking in projects (no matter if it is organizational or technical information) or feature locating within source code. In my opinion creating a solution that spans every aspect of onboarding is near impossible to achieve as it is such a broad topic. Thus, when creating a supportive digital solution, focusing on a distinct approach is crucial, e.g. by deciding on creating something that supports program comprehension or deciding on improving in-project communication, as vastly different intricacies are present depending on the area of interest. For each of these area of interests research already exists, as mentioned in chapter 2. Thus, an extension of the body of work is not necessary in the case of this question. Nonetheless, the literature discussed and knowledge gained is and was valuable as a foundation for the other parts of the thesis. Where it is important to gain new knowledge is to contextualize the answers the participants provided. This is necessary to understand how different personal approaches towards onboarding might influence the attitude towards playfulness in those approaches.

Concerning this research question, data was gathered on three mentioned different aspects (social, technical, organizational) and ideas from the participants towards aspects of onboarding that are of importance. Here the participants reported very different attitudes towards information in onboarding. P1 for example, as mentioned before, deemphasized the importance of contributors in onboarding\footnote{Coding Unit Nr. \textbf{3, 8} in \textit{Appendix \ref{append:coding}}}. Although it has to be noted, that in on-site projects P1 mentions that social contact and direct communication is important\footnote{Coding Unit Nr. \textbf{23} in \textit{Appendix \ref{append:coding}}}. As P1 mentions having not a lot of onboarding occations and 0-1 years of experience as a developer, this can be set in a context, where P1 might either see contributors not as direct colleagues or has more experience with on-site projects. Concerning these on-site projects, there is consensus between the participants that social links might be the most important in onboarding\footnote{Coding Units Nr. \textbf{53, 59, 76, 98, 123} in \textit{Appendix \ref{append:coding}}}.

A lot of technical onboarding considerations were mentioned by the participants as well. The participants reported mostly similar technical approaches to projects. P3, P8 and P9 all mentioned starting at the top-level of a project and then going into the details of single files and features of these projects\footnote{Coding Units Nr. \textbf{51, 56, 99, 122} in \textit{Appendix \ref{append:coding}}}. An important aspect of this initial approach is the access to the project itself, as for example P6 mentioned: \enquote{I try to get access to everything that I need (git, deployment server, ...)}\footnote{Coding Unit Nr. \textbf{75} in \textit{Appendix \ref{append:coding}}}. For P6, the \enquote{first step is always the documentation, if there are problems I google or search through the issue list}\footnote{Coding Unit Nr. \textbf{80} in \textit{Appendix \ref{append:coding}}}. This statement reinforces the importance of existing information on the project, be it within documentation or in 3rd-party forums. Here the project maintainers themselves can allow for a smoother onboarding experience by providing that -- if these are provided in a coherent manner, that kind of structure can be used within playful elements. Concerning the tools that are used throughout the onboarding process, mostly working directly within the \gls{ide} was mentioned\footnote{Coding Units Nr. \textbf{24, 79, 100, 122} in \textit{Appendix \ref{append:coding}}}. A notable addition to that cam from P8, was that the initial code navigation happens on Github (for projects hosted on Github at least)\footnote{Coding Unit Nr. \textbf{99} in \textit{Appendix \ref{append:coding}}}.

Regarding the organizational aspects, not as much input was given. There are two notable exceptions, though. P3 mentioned that it is very helpful to have a structured and well-organized issue list helps in identifying possible issues to start working on\footnote{Coding Unit Nr. \textbf{55} in \textit{Appendix \ref{append:coding}}}. P6 also brought up the fact that additional communication with non-developers might be worthwhile\footnote{Coding Unit Nr. \textbf{78} in \textit{Appendix \ref{append:coding}}}. This -- arguably also social aspect -- expands the onboarding further into non-technical areas, which have not been the focus up until this point.

To get back to answering the research question as a whole, this wide variety of aspects has to be contextualized. Overall, there are different areas in an onboarding process with each having different requirements and problems inherent to them. Thus, these changing requirements, depending on what is to be achieved, have to be kept closely in mind. In the case of this thesis, this meant for example, that the technical onboarding aspects were pronounced and considered in detail. for future approaches, focusing on the social aspects a different approach might be needed on the other hand.

\subsubsection*{(RQ4) How can the identified themes be applied in actual onboarding settings?}

For a subset of themes, this question can be considered answered already, as a possible application has been done as a part of the implementation of the technology probe. Concerning the playful themes and guidelines that were not implemented, possible applications can be subject to future research into that area. In addition to the documentation of the implementation process and thus the answer towards how to apply some of the themes, feedback on these applications was gathered throughout the study. These can inform how those applications could be improved or extended upon.

To be precise, broad input was provided on how existing features could be improved upon incrementally, but also where the current approach was experienced as lacking. Starting with the visualization of the project as a whole P8 mentioned an improved application through suggesting, that \enquote{there could be some moving dots from module to module, depending on how often the module gets imported into another module}\footnote{Coding Unit Nr. \textbf{101} in \textit{Appendix \ref{append:coding}}} and to adapt the visualization to grasp the importance of these modules\footnote{Coding Unit Nr. \textbf{97} in \textit{Appendix \ref{append:coding}}}. This could achieve a more detailed understanding of the underlying project through leveraging more detailed data from within the project itself. Integrating more data within these visualizations was also mentioned by other participants. P3 mentioned that non-technical information should have been included as well, such as the overall functions of the project itself\footnote{Coding Unit Nr. \textbf{43} in \textit{Appendix \ref{append:coding}}}. P6 also proposed presenting more information within the visualization, by proposing a visualization of the project organization structure together with suggestions of who to communicate with\footnote{Coding Unit Nr. \textbf{82} in \textit{Appendix \ref{append:coding}}} and a stronger connection to the project's source\footnote{Coding Unit Nr. \textbf{69} in \textit{Appendix \ref{append:coding}}}. As of smaller improvements to the implemented features, P1 mentioned adding more images and icons to the information on revealable objects\footnote{Coding Unit Nr. \textbf{10} in \textit{Appendix \ref{append:coding}}} and described how the reveal mechanic could be made more enticing, by increasing depth perception through using different colors\footnote{Coding Unit Nr. \textbf{16} in \textit{Appendix \ref{append:coding}}}.

Overall, most of the participants interacted closely with the implemented features and therefore the application of playful themes. Only P6 went through the probe very fast (as can be seen in \textit{Appendix \ref{append:log-aggregated}}) and did provide not that much feedback on game mechanic intricacies or detailed tasks, rather the feedback was targeted more towards the probe as a whole. Nonetheless, on all of the levels of detail, valuable input was gained on possible improvements to applications of play.

\subsubsection*{(RQ5) How do different kinds of software developers experience playful aspects within an onboarding context?}

The last of the (sub-)research questions is then specifically crafted to foray into a non-existing body of research -- the experience and attitude of developers towards playful elements in onboarding processes. The questions created for the probe mirror that and try to generate data on exactly that aspect of the research topic. The study ultimately succeeded in doing so and the participants brought forth many interesting aspects to this topic, that are going to be laid out in detail at this point. This allows for documenting different experiences to what was built by a set of participants with different experiences and professions.

P1 had a very clear opinion on playful elements for themselves and stated \enquote{this kind of visualization isn't the right thing to learn about a coding project. At least not for me.}\footnote{Coding Unit Nr. \textbf{20} in \textit{Appendix \ref{append:coding}}} P1 underlines that by giving an opinion on playful elements in onboarding processes as a whole. More specifically P1 mentions urgency and time available to get to know a new project and states that in time-sensitive situations playful elements hinder effective onboarding\footnote{Coding Unit Nr. \textbf{18, 19, 30} in \textit{Appendix \ref{append:coding}}}. P1 even describes it as potentially \enquote{frustrating}\footnote{Coding Unit Nr. \textbf{19} in \textit{Appendix \ref{append:coding}}}. What P1 also describes is the need for the right time and mindset necessary for a playful exploration of a project, where parts of a project are explored that might not be connected directly to the work done at the moment \footnote{Coding Unit Nr. \textbf{31} in \textit{Appendix \ref{append:coding}}}. An important aspect that P1 also mentions is that playful exploration should be done at a pace individual to the respective developer \footnote{Coding Unit Nr. \textbf{20} in \textit{Appendix \ref{append:coding}}}.

P3's attitude towards playful elements stands in contrast to P1's statements -- at least regarding the general opinion on those playful elements. P3 specifically brings up that a visualization coupled with playful exploration techniques might be able to provide a good \textit{first} overview of a project \footnote{Coding Unit Nr. \textbf{50, 60} in \textit{Appendix \ref{append:coding}}}.

P6 shares these sentiments with specifically pointing out that the playful elements and the exploration could provide a benefit to get an overview of software development projects\footnote{Coding Unit Nr. \textbf{72} in \textit{Appendix \ref{append:coding}}}. P6 acknowledges that this is not true at all times, though, by mentioning the inefficiency of such an approach in day-to-day work\footnote{Coding Unit Nr. \textbf{83} in \textit{Appendix \ref{append:coding}}} and that technical onboarding might better be done through \gls{ide}s and the such\footnote{Coding Unit Nr. \textbf{74} in \textit{Appendix \ref{append:coding}}}. P8 gives similar feedback, with mentioning the possible inefficiencies -- largely caused by the \textit{randomness} of the revealable objects -- of a playful approach\footnote{Coding Unit Nr. \textbf{96} in \textit{Appendix \ref{append:coding}}}. However, P8 also found aspects of the approach helpful, like pointing out potential issues of the underlying project\footnote{Coding Unit Nr. \textbf{89} in \textit{Appendix \ref{append:coding}}}, although that statement is not directly related to a playful approach, rather on the data extracted from the underlying project. P9 points out similar sentiments, albeit targeted more towards projects where getting an overview from the outside just by investigating the source code files might be hard to achieve\footnote{Coding Unit Nr. \textbf{121} in \textit{Appendix \ref{append:coding}}}.

To summarize these findings, there are mixed results on developer experience and attitude towards playful elements in onboarding. While there are statements towards the inefficiency of playful elements, the usage of these elements for getting initial overviews and exploring parts of the project that are not directly worked with, was described positively. Overall the participants reported no fundamental resentment towards play in onboarding processes, but mentioned important considerations and possible downsides of such approaches.

Some of those important considerations are connected to the context in which onboarding takes pace. Aspects of this were mentioned by multiple participants based on the question of playful elements within a work environment and open-source projects. P3 and P8 for example mention that in on-site working environments the access to other project members and experts can be significantly better. Thus direct in-person communication -- which is preferable for the participants -- is easier to achieve. Therefore, both P3 and P8 mention that an automated playful introduction should be more valuable when this in-person communication is not feasible\footnote{compare Coding Units Nr. \textbf{61, 62, 107, 109} in \textit{Appendix \ref{append:coding}}}. P6 and P8 also mention the motivation behind working on open source projects (\enquote{out of interest}\footnote{Coding Unit Nr. \textbf{84} in \textit{Appendix \ref{append:coding}}} or in a \enquote{self-motivated}\footnote{Coding Unit Nr. \textbf{108} in \textit{Appendix \ref{append:coding}}} way), which could increase the openness towards exploring playfully to learn and gain knowledge on a personal level -- rather than focusing on only what is needed to work on specific tasks. P6 also points out that playful approaches might be valuable for young developers or people starting out with development\footnote{Coding Unit Nr. \textbf{85} in \textit{Appendix \ref{append:coding}}}. Lastly P9, adds a another contextual consideration not strictly confined to open source projects but important nonetheless. P9 mentions that in large software projects that are hard to grasp, playful elements and exploration techniques could provide value\footnote{Coding Units Nr. \textbf{121, 124} in \textit{Appendix \ref{append:coding}}}. P9 also specifically points out that this could be helpful for projects built with programming languages the developer is not proficient in, but wants to learn for themselves\footnote{Coding Unit Nr. \textbf{44} in \textit{Appendix \ref{append:coding}}}.

\subsubsection*{(RQ1 -- Main RQ) How and which theories \& concepts from research on play can be used in the onboarding process of software development projects and how are developers experiencing them?}

After answering all the sub-research questions an answer for the overall main research question has to be crafted. Such an answer requires considering the answers to all the previous RQs, additional information from literature, but also all the explorative information generated throughout the study.

In order to use that information on an answer to this question the question has to be split up into its distinct parts: The playful theories \& concepts on one hand and the developers experience towards them. Concerning the playful theories \& concepts in onboarding processes a lot of results have already been generated through the explorative research that happened. This is now enriched with the analyzed study data, where possible. Regarding the second part of the question, this mostly was answered through RQ5.

As said, a lot of different possibilities of which playful theories \& concepts could be applied in an onboarding process has already been explored, which is described in detail in the answer to RQ2 and RQ4. During the analysis of the participant's answers multiple new ideas possibly worth exploring were mentioned. These can be summarized as follows:

\begin{itemize}
  \item{Creating a coherent \textit{quest line} with communicated sub goals to provide a diverse set of actions and mechanics to follow\footnote{Coding Units Nr. \textbf{26, 27, 33, 36} in \textit{Appendix \ref{append:coding}}}}
  \item{Make the participants solve problems, where the solution has to be found by the participants\footnote{Coding Unit Nr. \textbf{32} in \textit{Appendix \ref{append:coding}}}}
  \item{Set up a competition between project members to create additional motivation to find certain elements\footnote{Coding Unit Nr. \textbf{63} in \textit{Appendix \ref{append:coding}}}}
  \item{Explore a story with command line operations\footnote{Coding Unit Nr. \textbf{28} in \textit{Appendix \ref{append:coding}}}}
  \item{Place the companion directly within the working context, e.g. within an \gls{ide}\footnote{Coding Unit Nr. \textbf{81} in \textit{Appendix \ref{append:coding}}}}
  \item{Adding more common game mechanics, like fighting enemies (e.g. software bugs)\footnote{Coding Units Nr. \textbf{103, 105} in \textit{Appendix \ref{append:coding}}}}
  \item{Visualizing the project history\footnote{Coding Unit Nr. \textbf{106} in \textit{Appendix \ref{append:coding}}}}
  \item{Creating a character that follows the player within the game world\footnote{Coding Unit Nr. \textbf{112} in \textit{Appendix \ref{append:coding}}}}
  \item{Implement constraints to the explorable area, so that the player can explore these after each other while gaining knowledge in the process\footnote{Coding Unit Nr. \textbf{113} in \textit{Appendix \ref{append:coding}}}}
\end{itemize}

These ideas on playful elements provided by the participants complement what was found in literature already and present further options to explore and implement in the future.

Ultimately, an extensive list of possible design guidelines and playful themes has been created through carefully examining research in play and on software development onboarding. This list has been developed through following different perspectives onto play throughout the history on research into it. This was enriched with valuable input (shown above) from software developers generated through a study. Overall, this non-exhaustive list covers the question of \textit{which} playful elements could be used in an onboarding context. Concerning the how of this main research question in regards to the playful elements, this was answered through implementing a subset of this list and discussed through feedback by the study participants. Lastly, the question of how software developers experience such elements was answered through the study data as well. A critical discussion of these results is subject to discussion in the next chapter.