\section{Conclusion}

At this point, a theoretical foundation for the research topic at hand has been built, a study was designed, executed and analyzed, and its results were critically reflected upon. During the course of these actions knowledge, data and digital artifacts have been created and an attempt at answering the research questions was made. Valuable insights by developers were collected on their attitude towards playful elements in onboarding projects and initial feedback on possible digital implementations based on literature. To conclude this thesis, its specific contribution(s) are summarized once more and an outlook for future research is given.

\subsection{Contribution}

Within the introduction, the desired contribution was already described -- a combination of theoretical guidelines, a digital artifact and empirical findings researched through that artifact. These contributions could be created throughout the study. The theoretical guidelines were derived from literature on play and playfulness and are laid out in detail in chapter 5.1 with the lead-up described in chapter 2.4 and a graphical visualization shown in \textit{Appendix \ref{append:probe-design}}. These guidelines serve as an element of contribution by themselves but also serve as the foundation for the digital artifact crafted from it -- the interactive part of the technology probe.

\subsection{Outlook}

% Look ahead -> What comes after this thesis

% 1. Further development concerning what was created in this thesis

% 2. Further questions arising from this thesis

% 2.1 In which areas of onboarding could playful elements also be used?
% 2.1.1 How could playful elements support building and discovering social structures in onboarding processes?

% 2.2 Observing developers onboarding in-situ, gathering more data on that to find out at which points of the process, maybe even haptic and tangible games could be introduced (Beyond digital tools, focusing on social strctures)

% 2.3 Which specific benefits could playful elements bring to the table in onboarding? -> Quantitively researching benefits of specific solutions