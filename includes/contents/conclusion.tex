\section{Conclusion}

At this point, a theoretical foundation for the research topic at hand has been built, a study was designed, executed, and analyzed, and its results were critically reflected upon. During the course of these actions, knowledge, data, and digital artifacts have been created and an attempt at answering the research questions was made. Valuable insights by developers were collected on their attitude towards playful elements in onboarding projects and initial feedback on possible digital implementations based on literature. To summarize the results of the research done within this study, it can be said, that there is potential for the application of playful elements in onboarding processes. Most of the participating developers reported that they would be open towards such elements and provided valuable feedback on possible, additional elements and the ones implemented within the technology probe. An important distinction discovered in research is the fact that the participants reported that direct communication and social links are preferable to playful approaches in onboarding processes. Where this direct communication is complicated, other solutions are looked towards -- for example in remote project structures with high project member turnover. Multiple participants reported that this could be the case for open source projects. Playful approaches might be even more beneficial in this case, as developers working on open source projects often do so based on intrinsic motivation. This could make these developers more open towards a possibly less efficient but more enjoyable approach to onboarding. This concludes a summary of the study results of this thesis. These are not the only contribution though, therefore all the specific contribution(s) are summarized once more and an outlook for future research is given hereafter.

\subsection{Contribution}

Within the introduction, the desired contribution was already described -- a combination of theoretical guidelines, a digital artifact, and empirical findings researched through that artifact. These contributions could be created throughout the study. The theoretical guidelines were derived from literature on play and playfulness and are laid out in detail in chapter 5.1 with the lead-up described in chapter 2.4 and a graphical visualization shown in \textit{Appendix \ref{append:probe-design}}. These guidelines serve as an element of contribution by themselves but also serve as the foundation for the digital artifact crafted from it -- the interactive part of the technology probe. This digital artifact was created in order to \enquote{facilitate new insights or explarations} \cite[p. 2]{wobbrock2016research} as the artifact itself but more importantly through its usage as a technology probe. The detailed documentation of the implementation process, as well as the final interactive probe, is laid out in chapter 4 and the respective appendices mentioned throughout the thesis. This technology probe then generated the foundation for the third contribution of this thesis -- empirical data on the artifact and its implementation but also on how developers experience playful elements in onboarding in general. The empirical contribution, in this case, is the analyzed and interpreted set of data discussed in detail in chapter 5. This concludes the summary of contributions this thesis developed. As mentioned the specifics of these contributions are documented in detail in the respective chapters.

\subsection{Outlook}

After summarizing the contributions made throughout the thesis, at last, an outlook into possible future research is given. This includes possible research questions for future research, as well as specific opportunities and problems in the field to tackle in the future.

Regarding future research following this thesis, multiple possible research questions arise from what was found. One of them is the question of where playful elements could also be integrated into onboarding processes -- specifically on different onboarding areas. This could mean introducing playful elements to low-level program comprehension, or in contrast focusing on organizational structures surrounding a project. Other areas worth inspecting in greater detail would be to observe on-site onboarding processes in-situ, gathering data on how a project team organizes the onboarding of new members. This could allow for an additional angle on how to support this process as well, with tangible games and collective play as possible means to improve social aspects of onboarding. Concerning the results of the features of the technology probe, there is ample opportunity to craft more detailed prototypes supporting open source projects specifically. These in turn could even be quantitively analyzed and compared to traditional approaches in order to deduct reliable statements on improvements of the process.

Ultimately, this thesis acts as a first step into an intersection of research areas, where little previous research has happened. Thus, there are even more opportunities, than listed above, to extend but also to critically reflect what was found in the course of this thesis.