\section{Introduction}


\blockquote{
  \textit{Note for feedback}
  \textbf{This is going to be in this part:}

  Introducing what will be researched and the research fields involved.

}

\subsection{Motivation}

\blockquote{
  \textit{Note for feedback}
  \textbf{This is going to be in this part:}

  \textit{This is taken from the expose and is going to be reworked during the course of writing the thesis}

  With a projected growth of the number of software developers by about 4.8 million to 28.7 million in 2024 \cite{evans2019developers} it is paramount to research how these new developers can be onboarded in an ever-increasing number of software development projects. With an increase in the number of tools available as well, like packages and micro-packages in many ecosystems (e.g. Javascript \cite{kula2017impact}). As many of these packages can be freely combined in development projects, the complexity of these projects increases as well. Therefore it is of utmost importance to ease the onboarding of the increasing number of developers, within the increasing number of software projects with the increased complexity. Improvements in this area could reap real benefits like a decrease in onboarding time or increase in developer productivity and morale can directly affect the success of the respective of projects. While there are tools to help with that, as I am going to lay out later, these often require to already be within the project’s code and find your way from a specific point in code rather than helping holistic onboarding in projects. This is especially important for nascent and new software developers that do not know their way around a specific codebase but rather have to explore the project as a whole. I believe that theories, systems, and design patterns from Play Theory can help with that. This area of research is often used to create an enjoyable experience and provides methods for users to let them explore environments by themselves. Miguel Sicart for example states that play is a way of exploring the world \cite[p. 3]{sicart2014play} and this could be applicable on a smaller project-level as well. Therefore using these approaches could potentially lead to a more effective, more enjoyable way of onboarding software developers. This is why my master’s thesis is going to be all about: “Playful exploration in Software Development Projects”

  Special attention will be given to the emotional side of software development (\cite{graziotin2017unhappiness,graziotin2018happens,ortu2016emotional}) and if and how Play could lead to emotional benefits and a more enjoyable onboarding experience.

}

\subsection{Research Questions \& Contribution}

\blockquote{
  \textit{Note for feedback}
  \textbf{This is going to be in this part:}

  These are the research questions as of now:

  \begin{itemize}
    \item{\textbf{RQ1}: How can findings from play theory be used to in developer onboarding contexts in software projects?}
    \item{\textbf{RQ2}: Which playful exploration strategies, patterns, systems, and theories could lend themselves well to evolve software development projects into a space that enables play?}
    \item{\textbf{RQ3}: What are the technical, social and personal intricacies of an onboarding context in software development and how do they influence the onboarding experience?}
    \item{\textbf{RQ4}: How can the pre-discovered strategies, patterns, systems and \& theories be applied in actual onboarding settings?}
    \item \item{\textbf{RQ5}: How do different kinds of software developers experience playful aspects within an onboarding context?}
  \end{itemize}

  The contribution of this thesis is going to consist of two parts where one of those builds upon the other. Firstly, empirical, qualitative data is going to be generated during problem-centered interviews with developers. This dataset is going to be analyzed and together with the literature research part a list of possible informed implications and insights for the creation of an artifact is going to be put together. Finally, a digital artifact (comparably to \cite{wobbrock2016research}) is going to be created as the other main contribution. It is going to be build upon what was researched before and either tries to create a coherent solution informed by the research results or consist of multiple small interactive prototypes each prototyping a single insight. This is going to be decided after gathering the list of implications.

  Additionaol or alternative contribution are recommendations for using play in the investigated context, what to look for, what to investigate, ...

}